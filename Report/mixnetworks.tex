\subsection{Mix Networks}

\subsubsection{Overview}
http://www.rsa.com/rsalabs/staff/bios/ajuels/publications/universal/Universal.pdf

One purpose of mix networks, or mixnets, is to provide untraceability
to its users. A mixnet may, for example, take as input a list of
encrypted messages of different origins. These messages pass through
the mixnet and is output decrypted and in a randomized order. This
property may be used to enable anonymous voting systems.

Different types of mixnets exist. There are decryption mixnets and
reencryption mixnets. This report treats the latter type. A
reencryption mixnet consists of a number of servers which sequentially
process the messages and reencrypts the list of messages and outputs
them in a randomized order. After passing through all servers, the
list of ciphertexts is decrypted and the result is a list of the
messages in random order. It should be infeasible to deduce from where
each element came.

One use of mixnets is in the context of electronic voting systems. An
electronic election can be performed by the use of a reencryption
mixnet \\
(http://courses.csail.mit.edu/6.897/spring04/L17.pdf):
\begin{enumerate}
\item The mix servers prepare the mix-net by generating public and
  secret keys.
\item Each voter encrypts his vote and appends it to a public list of
  encrypted votes.
\item In sequential order each mix server takes as input the list of
  encrypted votes, reencrypts and outputs them in a randomized order,
  replacing the previous list of encrypted votes.
\item After all mix servers have processed the list, each vote is
  jointly decrypted and posted on a bulletin board making the outcome
  of the election universally available.
\end{enumerate}

Notice that the reencryption step is necessary before the actual
mixing as if omitted, the mixing would merely scramble the list of
original cryptotext, providing no untraceability at all.

\subsubsection{El Gamal Mixnets}

Most reencryption networks use some variant of the El Gamal
Cryptosystem (http://courses.csail.mit.edu/6.897/spring04/L17.pdf),
since the homomorphic property of the El Gamal encryption allowes
reencryption.

A mixnet based on the El Gamal Cryptosystem consists of $k$ mixnet
servers mixing the votes of $n$ voters. Suppose the underlying group
is $G_q$ of prime order $q$ and with generator $g \in G_q$. The mixnet
works as follows \\
(http://courses.csail.mit.edu/6.897/spring04/L18.pdf)

\begin{enumerate}
\item A public key $pk = (g,y) = (g, g^x)$ is generated (infoga text om key distribution?)
\item Each voter $j$ encrypts his vote $m_j$ to create $c_{j,0} =
  \mathrm{Enc}_{pk}(m_j) = (g^{r_j},my^{r_j})$ for some random $r_j
  \in \mathbb{Z}_q$.  A list of all encrypted votes $c_0 = \left(
  c_{1,0}, \hdots c_{n,0}\right)$ is created.
\item For each mix server $i \in \{1,\hdots, k\}$, given the input
  $c_{i-1}$, a random permutation $\pi _i$ is chosen and a list 
  $$ 
  c_i =\left(\mathrm{ReEnc}_{pk}(c_{\pi_i(1),i-1}), \hdots,
  \mathrm{ReEnc}_{pk}(c_{\pi_i(n), i-1})\right) =
  $$
  is returned.
\item The final list $c_k$ is decrypted using the secret key $sk = x
  \in \mathbb{Z}_q$ (infoga text om key distribution?) to produce
  the output list
 $$ 
  (m_{\pi (1)}, \hdots , m_{\pi (n)}) =
  \left(\mathrm{Dec}_{sk}(c_{k,1}), \hdots, \mathrm{Dec}(c_{k,n})\right)
  $$
  
  for a permutation $\pi = \pi_k \circ \hdots \circ \pi_1$.
\end{enumerate}

The result of the election may now be computed while the origin of the
individual votes is unknown. Remark that all encryptions
$\mathrm{Enc}_{pk}$ are performed with some randomness $r \in
\mathbb{Z}_q$.

\subsubsection{Verification}
There are some problems related to electronic voting using
mixnets. One issue is that the mix servers may or may not execute
their part of the mixnet properly. For, example dishonest servers
could completely change the outcome of the election by replacing the
true votes with their own. A first solution may be to make sure that
every server is reliable. This is however difficult, as even an honest
party providing a computer acting as mix server may be affected by a
virus of some sort. Another and more feasible solution is to allow
verification by external parties.

A verifiable 





