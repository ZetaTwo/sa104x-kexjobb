\subsection{Mix Networks}

There are different kinds of mix networks. Here, only reencryption
mix-nets will be treated - first on a general level followed by a
description of a certain type of reencryption mix-net based on the El
Gamal cryptosystem. In any mix-net used in the context of electronic
elections, the need for verification of correct execution is
essential. How this can be obtained is explored in the end of the
section.

\subsubsection{Overview}
One purpose of mix networks is to provide untraceability to its
users. A mix-net may, for example, take as input a list of encrypted
messages of different origins. These messages pass through the mix-net
and is output decrypted and in a randomized order \cite{mixnets}. This
property may be used to enable anonymous voting systems \cite{mixnets2}.

Different types of mix-nets exist. There are decryption mix-nets and
reencryption mix-nets. A reencryption mix-net consists of a number of
servers, henceforth the mix servers, which sequentially process the
messages by reencrypting the list of messages and outputting them in a
randomized order \cite{mixnets}. After passing through all servers,
the list of ciphertexts is decrypted and the result is a list of the
messages in random order. It should after decryption be infeasible to
deduce the position of each element in the initial list. We define
this as \emph{untraceability} \cite{mixnets2}. This provides the
voters with anonymity.

One use of mix-nets is in the context of electronic voting systems. An
electronic election can be performed by the use of a reencryption
mix-net as follows \cite{electronicvoting} \\
\begin{enumerate}
\item The mix servers prepare the mix-net by generating public and
  secret keys.
\item Each voter encrypts his vote with the public key and appends it
  to a public list of encrypted votes.
\item In sequential order each mix server takes as input the list of
  encrypted votes, reencrypts and outputs them in a randomized order,
  replacing the previous list of encrypted votes.
\item After all mix servers have processed the list, each vote is
  jointly decrypted and posted on a bulletin board making the outcome
  of the election universally available.
\end{enumerate}

Notice that the reencryption step is necessary before the actual
mixing as if it were omitted, the mixing would merely scramble the
list of original cryptotexts, providing no untraceability at all.

\subsubsection{El Gamal Mix-Nets}

A common choice for reencryption mix-nets is to use some version of
the El Gamal cryptosystem \cite{electronicvoting}, since its
homomorphic property allows reencryption. A mix-net based on the El
Gamal cryptosystem consists of $k$ mix-net servers mixing the votes of
$n$ voters. Suppose the underlying group is $G_q$ of prime order $q$
and with generator $g \in G_q$. The mix-net works as follows
\cite{electronicvoting2}\\

\begin{enumerate}
\item An El Gamal public key $pk = (g,y)$ is generated.
\item Each voter $j$ encrypts his vote $m_j$ to create $c_{j,0} =
  \mathrm{Enc}_{pk}(m_j) = (g^{r_j},my^{r_j})$ for some random $r_j
  \in \mathbb{Z}_q$.  A list of all encrypted votes $c_0 = \left(
  c_{1,0}, \hdots c_{n,0}\right)$ is created.
\item For each mix server $i \in \{1,\hdots, k\}$, given the input
  $c_{i-1}$, a random permutation $\pi _i$ is chosen and a list 
  $$ 
  c_i =\left(\mathrm{ReEnc}_{pk}(c_{\pi_i(1),i-1}), \hdots,
  \mathrm{ReEnc}_{pk}(c_{\pi_i(n), i-1})\right) =
  $$
  is returned.
\item The final list $c_k$ is decrypted using the secret key $sk = x
  \in \mathbb{Z}_q$ to produce the output list
 $$ 
  (m_{\pi (1)}, \hdots , m_{\pi (n)}) =
  \left(\mathrm{Dec}_{sk}(c_{k,1}), \hdots, \mathrm{Dec}(c_{k,n})\right)
  $$
  
  for where $\pi$ is a permutation $\pi = \pi_k \circ \hdots \circ
  \pi_1$.
\end{enumerate}

 The result of the election may now be computed while the origins of
 the individual votes are unknown. Remark that all encryptions
 $\mathrm{Enc}_{pk}$ are performed with some randomness $r \in
 \mathbb{Z}_q$.

\subsubsection{Verification}
There are some problems related to electronic voting using
mix-nets. One issue is that the mix servers may or may not execute
their part of the mix-net properly. For example dishonest servers
could completely change the outcome of the election by replacing the
true votes with their own \cite{electronicvoting}. A first solution
may be to make sure that every server is reliable. This is difficult,
however, since there is always a risk that a server is
compromised. Another and more feasible solution is to allow
verification by external parties.

The verification of execution relies on a concept called
zero-knowledge proof. A zero-knowledge proof is a cryptographic
protocol that can be used by one party, the prover, to prove to
another party that some statement is true. The greatest benefit of
using this protocol is that the prover can convince the other party of
the truthfulness of the statement without actually revealing any
additional information \cite{shoup}.

It is possible to create a zero-knowledge proof based on the discrete
logarithm problem \cite{shoup}. In this interactive scheme the prover
is able to prove that he or she possesses some secret discrete
logarithm of an element in a group and this will be done without
actually revealing the secret. Only a transcript, containing
communication during the scheme, is generated. In fact, the scheme
does not reveal any additional information because any party is able
to produce a correct looking transcript on their own, which means that
reusing (showing it someone else) an old transcript does not prove
knowledge of the secret.

When it comes to mix-nets, in order to ensure correct execution, one
can use zero-knowledge proofs to prove that every party shuffles their
input ciphertexts and reencrypts these ciphertexts. The shuffling part
may be proven to be correct by a zero-knowledge proof that proves that
two lists of ciphertexts are permutations of each other
\cite{terelius}.

It is also important to consider the possibility that one corrupt mix
server may reveal information that should not be revealed, for example
revealing the origin of some vote \cite{electronicvoting}. This is a
problem in the El Gamal mix-net described above, since if all parties
were using the same public key $pk$, all parties would also possessed
the same secret key.

In order to achieve untraceability, the mix-net servers need to
generate different public keys and still remain able to jointly
decrypt the output list of ciphertexts. The El Gamal cryptosystem
provides protocols for distributed key generation and decryption by
several parties \cite{wikstrom1}. The details of these protocols will
be omitted, but the basic principle is the following.

A beforehand specified threshold number of mix-net servers
independently generate their own secret keys. The $l$th mix-net server
generates $sk_l = x_l \in \mathbb{Z}_q$, from which a partial public key
$pk_l = (g,y_l) = (g,g^{x_l})$ can be derived. A joint public key is
then generated as $pk = \prod_l y_l$ and the corresponding secret key
$sk = \prod_l x_l$. After each server has processed the ciphertext
list, the list is jointly decrypted using a certain procedure. Omitted
in this description is the fact that the secret keys are actually
shared among all mix-net servers in such a way that no subset of the
servers smaller than the threshold number is able to reveal
information about the voting before completion of execution. This
protocol with a suitably chosen threshold values makes it possible to
correctly execute the mix-net even when the mix-net contains dishonest
parties \cite{wikstrom1}.


