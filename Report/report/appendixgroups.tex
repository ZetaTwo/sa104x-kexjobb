\clearpage
\section{Appendix A - Group Theory}

A group $(G, \cdot)$ is a set $G$ and a binary operation $\cdot: G \to G$, called
multiplication, that satisfies the four group axioms:
\begin{enumerate}
\item The product of any two elements of $G$ is in $G$
$$\forall a,b \in G, \, a\cdot b \in G$$

\item Multiplication is associative 
$$
\forall a,b,c \in G, \, (a\cdot b)\cdot c = a \cdot (b \cdot c)
$$ 
\item There exists a unique identity element in $G$
$$
\exists! e \in G \text{ such that } \forall a \in G, \, a \cdot e = e \cdot a = a
$$
\item Every element in $G$ has an inverse
$$
\forall a \in G, \, \exists b \in G \text{ such that } a \cdot b = b \cdot a = e
$$
\end{enumerate}

When the operation of the group $(G, \cdot)$ is understood from the
context, one often abbreviates the notation and calls $(G, \cdot)$
simply $G$.

Oftentimes one wants to multiply an element to itself a number of
times. This is called exponentation. Let $g \in G$ and
$x \in \mathbb{Z}$, then $g$ multiplied to itself $x$ times
$$
g \cdot ... \cdot g = g^x
$$

is called $g$ raised to the power of $x$. The usual laws of
exponentation holds in a group.

The number of elements in a group $G$ is called the order of the
group. If the order of $G$ is finite, $G$ is said to be finite. If
$x \in \mathbb{Z}$ is the smallest integer such that $g^x = e$, then
$x$ is called the order of the element $g \in G$. Notice that if
$y,y' < x$, $y \neq y'$, then $g^y \neq g^{y'}$ as otherwise
$$
g^y = g^{y'} \Rightarrow e = g^{y'} \cdot g^{-y} = g^{y' - y}
$$
where $0 < y' - y < x$ and hence $x$ is not the order of $g$, which is
a contradiction.

If the order of $G$ is $q < \infty$ and there exists an element $g \in
G$ such that the order of $g$ is $q$, then $G$ is said to be cyclic
with $g$ as a generator. The motivation behind this naming, is that
any element $h \in G$ is equal to $g^x$ for some $x \le q$, so $g$
generates the whole group.



Multiplicative subgroups of $Z_n$
