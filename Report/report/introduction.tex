\section{Introduction}

The area of cryptography concerns the theory of secure
communication. For this to have any meaning we must define
"secure". This is not easy since the most strict definition is not
practically feasible and is thus done on a case to case basis
depending on the context.

Originally cryptography was interchangeable with the encryption of
messages in order to allow secret communication between two parties
over an insecure channel. Early examples include various ciphers such
as the Caesar and the Vigenère ciphers. These primitive systems
achieve the most central property of a cryptographic system,
\emph{confidentiality}, i.e. only parties in possession of the key can
encrypt and decrypt the message \cite[p.~3]{hac}. This is also an example of a
\emph{symmetric encryption} which means that the same piece of
information, the key, is used for encryption and decryption.

One problem that arises when only confidentiality is considered is
that even though a third party cannot read the message, it can be
modified without the receiver noticing. One way to counter this is to
introduce some kind of "receipt" computed from the message and sent
along with the ciphertext. This receipt computed again by the receiver
to verify that the message is unaltered. This property is what we call
\emph{integrity}.

In 1976 Whitfield Diffie and Martin Hellman publiced the first paper
on \emph{public key cryptography} \cite[p.~2]{hac}. This was a new sort of
cryptographic construction that allowed different keys to be used for
encryption and decryption of a message. It was the first
\emph{asymmetric encryption} scheme.

Already at this point we have a very powerful set of tools for secret
communication. However, one thing which is not covered by traditional
public key cryptography is \emph{anonymity}. By design, two
communicating parties must have some kind of knowledge of each other
to be able to communicate. 

A cryptographic construction, called a mix network or mix-net,
facilitates the anonymity requirement by routing messages to a cluster
of nodes \cite[p.~1]{mixnets}. These nodes takes a number of messages
from any number of sources, shuffle them around and deliver them to
their destination. This description is very brief, partly because
there are different flavours of mix-net intended for different
purposes.

\subsection{Background}

The Verificatum mix-net is designed to take a number of encrypted
messages, ciphertexts, which have been encrypted with a
public key from the mix-net, shuffle them around and eventually
decrypt them. This can, for example, be used in an election. Every voter
encrypts a vote with the public key and posts it to the mix-net. The
mix-net takes all the encrypted votes, shuffles and decrypts
them. This way all votes are revealed and are possible to count without
revealing information about which vote belongs to which voter - 
an important property in elections.

Additionally we would like to be able to verify that the mix-net has
done its work correctly and not altered any votes. Therefore the
mix-net produces some extra data files during its execution. These
data files together with the input ciphertexts and the output
plaintexts can be run through a verification process that tells if the
process has been correctly performed. The neat thing about all this is
that it can securely tell whether this is true or false without
spilling any more information about the relation between the
ciphertexts and the plaintexts, i.e. voter anonymity is preserved.

\subsection{Goals and Scope}

The idea is that any third party that wants to check the result should
be able to implement a verifier for Verificatum. Wikström has created
a document describing this verifier \cite{wikstrom1}. We have, using
this document, implemented a verifier to test that the document is
complete with all the information needed for the
implementation. Because of time constraints, we have limited ourselves
to implement a working prototype with only the most common options
available, just sufficient to confirm that the specification contain
all the necessary parts to implement a verifier. This prototype could
later be expanded to facilitate for all features of the Verificatum
mix-net.
