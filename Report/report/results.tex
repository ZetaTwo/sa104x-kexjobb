\section{Results (TODO)}

\subsection{Performance}

TODO: Fixa detta om vi lyckas få en fungerande version.

\subsection{Comments on the Documentation}

The overall impression of the documentation was that it contained all
information needed to implement the verifier. There were, however,
errors affecting the execution of the verifier. Some of (beror på hur
långt vi kommer med implementeringen) the problems arising from these
errors were overcome after discussion with Wikström. See Appendix B
for a complete list with errors found and specific comments about the
report.

In the absolute beginning of the document, the reader is thrown into
details about the zero-knowledge proofs used in VMN. A more
gentle approach would be to introduce the VMN without
assuming too much acquaintance with it or provide the means for the
reader to do so on his or her own.

The background, including a description of a mix-net based on the El
Gamal cryptosystem, is clear and concise. After the background, the
document contains a list of manageable subtasks in order to facilitate
implementation. This list was appreciated as it gave someone
unacquainted with Verificatum ideas of suitable starting points.

Chapters 4 through 6 contain necessary and easily accessible
information. However, chapter 4 on byte trees and chapter 6 on
representation of arithmetic objects are closely related and could
advantageously be presented together while bringing up the
cryptographic primitives of chapter 5 afterwards. The part on deriving
group elements from random strings depends on chapter 5 and
consequently needs to be presented after. See Appendix C for a
clarification on these comments.

Chapter 6.6 on Marshalling Groups contains specific details regarding
the Verificatum software with strong connections to Java. This chapter
could be rephrased to only include information actually needed for
implementation of a verifier.

Lastly, the chapter on verification of Fiat-Shamir proofs relies
heavily on the derivation of group elements from random strings. By
moving chapter 7 to before the chapter on cryptographic primitives,
usage of the document will probably demand less page turns.

\subsection{Conclusion}

Regarding the programming it would have helped with having more layers of abstractions in the code. Specifically, some classes representing various mathematical objects would have made the code easier to maintain. We believe that a more solid understanding of the structure of VMN before we started programming would have helped in creating a better structure of the verifier. The choices of third party libraries were good and they all were easy to use in our project. This also makes amount of code which need to be written smaller.

The specification document for the verifier does include all the information required to write the verifier. However, as described earlier, the structure could be improved for greater readability. There is also some unnecessary information in the document. Lastly, the document could benefit from an improved description of the VMN so that one easier can get a better understanding of what approach to take when implementing the verifier.

We can conclude that the document contains all the information but we still weren't able to implement the verifier without test data to verify our code.
