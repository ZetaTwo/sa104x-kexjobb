\section{Results (TODO)}

\subsection{Performance}

TODO
We must do this. Must we?

\subsection{Comments on the Documentation}

The overall impression of the documentation was that it contained all
information needed to implement the verifier. There were, however,
errors affecting the execution of the verifier. Some of (beror på hur
långt vi kommer med implementeringen) the problems arising from these
errors were overcome after discussion with Wikström. See Appendix B
for a complete list with errors found and specific comments about the
report.

In the absolute beginning of the document, the reader is thrown into
details about the zero-knowledge proofs used in Verificatum. A more
gentle approach would be to introduce the Verificatum mix-net without
assuming to much acquaintance with it or provide the means for the
reader to do so on his or her own.

The background, including a description of a mix-net based on the El
Gamal cryptosystem, is clear and concise. After the background, the
document contains a list of manageable subtasks in order to facilitate
implementation. This list was appreciated as it gave someone
unacquainted with Verificatum ideas of suitable starting points.

Chapters 4 through 6 contain necessary and easily accessible
information. However, chapter 4 on byte trees and chapter 6 on
representation of arithmetic objects are closely related and could
advantageously be presented together while bringing up the
cryptographic primitives of chapter 5 afterwards. The part on deriving
group elements from random strings depends on chapter 5 and
consequently needs to be presented after. See Appendix C for a
clarification on these comments.

Chapter 6.6 on Marshalling Groups contains specific details regarding
the Verificatum software with strong connections to Java. This chapter
could be rephrased to only include information actually needed for
implementation of a verifier.

Lastly, the chapter on verification of Fiat-Shamir proofs relies
heavily on the derivation of group elements from random strings. By
moving chapter 7 to before the chapter on cryptographic primitives,
usage of the document will probably demand less page turns.

\subsection{Conclusion}

