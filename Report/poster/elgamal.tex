\documentclass[18pt,a4paper]{article}
\usepackage[utf8]{inputenc}
\usepackage{amsmath}
\usepackage{amsfonts}
\usepackage{amssymb}
\usepackage{parskip}

\begin{document}

\section*{El Gamal Cryptosystem}
The El Gamal cryptosystem is a public key cryptosystem based on the
computational difficulty of computing discrete logarithms in cyclic
groups. 

The cryptosystem is defined over a group $G_q$ of order $q$ with
generator $g$. The secret key $x$ is chosen randomly in $\mathbb{Z}_q$
and the public key $y$ is created as follows
$$
y = g^x
$$

Encryption of a plaintext $m \in G_q$ is done by choosing a random $s
\in \mathbb{Z}_q$ and computing
$$
(u,v) = (g^s, y^sm) \in G_q \times G_q
$$

Decryption of a ciphertext $(u,v) \in G_q \times G_q$ is achieved by
using the private key $x$ to compute
$$ 
u^{-x}v = (g^s)^{-x}y^sm = (g^x)^{-s}y^sm = y^{-s}y^sm = m
$$

The El Gamal Cryptosystem possesses a homomorphic property. This means
that the encryption of the product of two plaintext messages is the
same as the product of the individual encryptions of the
plaintexts. The randomnesses is replaced by the combined randomness of
the individual encryptions. By choosing one of the messages to the
identity element in the group one has obtained the ability to
reencrypt a particular ciphertext without knowing the original
plaintext nor the randomness. This property of a cryptosystem is
necessary if it should be used in a reencryption mix-net.

\section*{Security}

The security of the El Gamal cryptosystem relies on the Decisional
Diffie-Hellman assumption. The assumption is closely related to the
discrete logarithm in cyclic groups.

The discrete logarithm is a generalization of the usual logarithm to
groups. Let $b = g^a \in G_q$ where $a \in \mathbb{Z}_q$, then $a$ is
said to be the discrete logarithm of $b$ in the group $G_q$. There is
currently no known efficient classical algorithm that given $(G_q, g,
b)$ is able to calculate $a$ in a reasonable amount of time
(polynomial time). The discrete logarithm problem is thus considered
to be a hard problem.

The Decisional Diffie-Hellman assumption concerns a problem related to
the discrete logarithm. The assumption in a certain group $G_q$ means
that if $a,b,c \in \mathbb{Z}_q$ are chosen randomly, every efficient
algorithm, on input $g^a$, $g^b$ and $y \in \{g^{ab}, g^c\}$, is
unable to tell if $y = g^{ab}$ or $y = g^c$.

The security of the El Gamal cryptosystem relies on the Decisional
Diffe-Hellman assumption in finite cyclic groups $G_q$. This means
that the El Gamal cryptosystem is secure as long as the assumption is
true.

\end{document}
