\section{El Gamal Cryptography}

\subsection{Definition}
The El Gamal cryptosystem is defined over a group $G_q =
\left<g\right>$ of prime order $q$, generated by $g \in G_q$. A
private key $x \in \mathbb{Z}_q$ is chosen randomly and is used to
compute the public key $(g,y) \in G_q \times G_q$ where $y =
g^x$. 

Encryption of a plaintext $m \in G_q$ is done by choosing a random $s
\in Z_q$ and computing $ \mathrm{Enc}_{pk}(m,s) = (u,v) \in G_q \times
G_q$ where $u = g^s$ and $v = y^sm$. Decryption of a ciphertext $(u,v)
\in G_q \times G_q$ is achieved by using the private key $x$ to
compute $m = \mathrm{Dec}_{pk}(u,v) = u^{-x}v$.

One common choice of group to use in the El Gamal Cryptosystem is a
multiplicative subgroup $G_q \subset \mathbb{Z}_p^*$, where $p = kq +
1$, for some $k$. Another common choice is to use certain elliptic
curves, in which case one may obtain the same security with a smaller
group and hence a more space efficient implementation. (Källa)

\subsection{Security}
A cryptosystem is said to be semantically secure if any efficient
(probabilistic, polynomial time) algorithm cannot with non-negligible
probability distinguish between the encryption of two different
plaintexts.

Infoga fullständig definition också?

Let $b = g^a \in G_q$ where $a \in \mathbb{Z}_q$. Then $a$ is said to
be the discrete logarithm of $b$ in the group $G_q$. There is
currently no known efficient classical algorithm that given $(G_q, g,
b)$ is able to calculate $a$ in a reasonable amount of time
(polynomial time). The discrete logarithm problem is thus considered
to be a hard problem. (Källa)

The Decisional Diffie-Hellman assumption is an assumption concerning a
problem related to the discrete logarithm. If the assumption holds in
a certain group $G_q$, then if $a,b,c \in \mathbb{Z}_q$ are chosen
randomly, every efficient algorithm, on input $g^a$, $g^b$ and $y \in
\{g^{ab}, g^c\}$, is unable to tell if $y = g^{ab}$ or $y = g^c$.

The security of the El Gamal cryptosystem relies on the Decisional
Diffe-Hellman assumption in finite cyclic groups $G_q$. This means
that the El Gamal cryptosystem is secure as long as the assumption is
true. (Källa)

\subsection{Properties}
If $G_q$ is a group, then so is $G_q \times G_q$ with the group
operation defined as $(a,b)(c,d) = (a c, b d)$
for any $(a,b),(c,d) \in G_q \times G_q$. With this definition, the El Gamal
cryptosystem is a homomorphic cryptosystem. This means that for any
two messages $m_1, m_2 \in G_q$ and randomnesses $s_1, s_2 \in
\mathbb{Z}_q$
$$
 \mathrm{Enc}_{pk}(m_1, s_1)\mathrm{Enc}_{pk}(m_2, s_2) =
(g^{s_1}, y^{s_1}m_1)(g^{s_2},y^{s_2}m_2) =
$$
$$
= (g^{s_1 + s_2}, y^{s_1 + s_2}m_1m_2) = \mathrm{Enc}_{pk}(m_1m_2, s_1 + s_2)
$$

In particular, by choosing $m_1 = m$ and $m_2 = 1$ one obtains
$$
\mathrm{Enc}_{pk}(m, s_1) \mathrm{Enc}_{pk}(1, s_2) = \mathrm{Enc}_{pk}(m, s_1 + s_2)
$$

This homomorphic property of the El Gamal Cryptosystem may be used to
reencrypt an already encrypted message. If $s_1 \in \mathbb{Z}_q$ and
$s_2 \in \mathbb{Z}_q$ are both chosen with uniform randomness, then
$s_1 + s_2 \in \mathbb{Z}_q$ will be uniformly random as well
(utveckla + källa). So the distribution of ciphertexts encrypted once
will be indistinguishable from the distribution of ciphertexts that
have been reencrypted. (Källa)

For future convenience define
$$
\mathrm{ReEnc}_{pk}(c,s) = c \cdot \mathrm{Enc}_{pk}(1,s) 
$$

Infoga text om key distribution här?


\subsection{Generalization}
A generalization of the El Gamal Cryptosystem over a group $G_q$ can
be achieved by considering the plaintext group $M_w$ to be $G_q \times
.. \times G_q = G_q^w$ and the ciphertext group to be $C_w = M_w
\times M_w$ (källa: douglas doc). Encryption and decryption is done
componentwise and the group operation of $M_w$ will also be performed
componentwise. This is useful since it allows longer plaintexts to be
encrypted.
 
Förklara mer?
