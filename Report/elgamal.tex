\section{El Gamal Cryptography}

\subsection{Definition}
The El-Gamal cryptosystem is defined over a group $G_q =
\left<g\right>$ of prime order $q$, generated by $g \in G_q$. A
private key $x \in \mathbb{Z}_q$ is chosen randomly and is used to
compute the public key $(g,y) \in G_q \times G_q$ where $y =
g^x$. 

Encryption of a plaintext $m \in G_q$ is done by choosing a random $s
\in Z_q$ and computing $ \mathrm{Enc}_{pk}(m,s) = (u,v) \in G_q \times
G_q$ where $u = g^s$ and $v = y^sm$. Decryption of a ciphertext $(u,v)
\in G_q \times G_q$ is achieved by using the private key $x$ to
compute $m = \mathrm{Dec}_{pk}(u,v) = u^{-x}v$.

\subsection{Security}
Let $b = g^a \in G_q$ where $a \in \mathbb{Z}_q$. Then $a$ is said to
be the discrete logarithm of $b$ in the group $G_q$. There is
currently no known efficient classical algorithm that given $(G_q, g,
b)$ is able to calculate $a$ in a reasonable amount of time
(polynomial time). The discrete logarithm problem is thus considered
to be a hard problem. (Källa)

The security of the El Gamal cryptosystem relies on the difficulty of
discrete logarithm in finite cyclic groups $G_q$. This means that the
El Gamal cryptosystem is secure as long as no one is able to compute
the discrete logarithm in $G_q$ efficiently. (Källa)

\subsection{Properties}
The El Gamal cryptosystem is a homomorphic cryptosystem. This means
that for any two messages $m_1, m_2 \in G_q$ and randomnesses $s_1,
s_2 \in \mathbb{Z}_q$
$$
\mathrm{Enc}_{pk}(m_1, s_1)\mathrm{Enc}_{pk}(m_2, s_2) = (g^{s_1}, y^{s_1}m_1)(g^{s_2},y^{s_2}m_2) =
$$
$$
= (g^{s_1 + s_2}, y^{s_1 + s_2}m_1m_2) = \mathrm{Enc}_{pk}(m_1m_2, s_1 + s_2)
$$

By choosing $m_1 = m$ and $m_2 = 1$ one obtains
$$
\mathrm{Enc}_{pk}(m, s_1) \mathrm{Enc}_{pk}(1, s_2) = \mathrm{Enc}_{pk}(m, s_1 + s_2)
$$

This homomorphic property of the El Gamal Cryptosystem may be used to
reencrypt an already encrypted message. If $s_1 \in \mathbb{Z}_q$ and
$s_2 \in \mathbb{Z}_q$ are both chosen with the uniform randomness,
then $s_1 + s_2 \in \mathbb{Z}_q$ will be uniformly random as well
(utveckla + källa).

\subsection{Generalization}
A generalization of the El Gamal Cryptosystem over a group $G_q$ can
be achieved by considering the plaintext group $M_w$ to be $G_q \times
.. \times G_q = G_q^w$ and the ciphertext group to be $C_w = M_w
\times M_w$ (källa: douglas doc). Encryption and decryption is done
componentwise and the group operation of $M_w$ will also be performed
componentwise. 
 
Förklara mer?
