\section{El Gamal Cryptography}

\subsection{Definition}
The El-Gamal cryptosystem is defined over a group $G_q =
\left<g\right>$ of prime order $q$, generated by $g \in G_q$. A
private key $x \in \mathbb{Z}_q$ is chosen randomly and is used to
compute the public key $(g,y) \in G_q \times G_q$ where $y =
g^x$. 

Encryption of a plaintext $m \in G_q$ is done by choosing a random $s
\in Z_q$ and computing $(u,v) \in G_q \times G_q$ where $u = g^s$ and
$v = y^sm$. Decryption of a ciphertext $(u,v) \in G_q \times G_q$ is
achieved by using the private key $x$ to compute $m = u^{-x}v$.

\subsection{Security}
Let $b = g^a \in G_q$ where $a \in \mathbb{Z}_q$. Then $a$ is said to
be the discrete logarithm of $b$ in the group $G_q$. There is
currently no known efficient classical algorithm that given $(G_q, g,
b)$ is able to calculate $a$ in a reasonable amount of time
(polynomial time). The discrete logarithm problem is thus considered
to be a hard problem. (Källa)

The security of the El Gamal cryptosystem relies on the difficulty of
discrete logarithm in finite cyclic groups $G_q$. This means that the
El Gamal cryptosystem is secure as long as no one is able to compute
the discrete logarithm in $G_q$ efficiently. (Källa)

\subsection{Properties}
The El Gamal cryptosystem is a homomorphic cryptosystem. This 




Generalization
