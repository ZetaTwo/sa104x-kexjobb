\documentclass[10pt,a4paper]{article}
\usepackage[utf8]{inputenc}
\usepackage{amsmath}
\usepackage{amsfonts}
\usepackage{amssymb}

\author{Carl Svensson carlsven@kth.se \& Erik Larsson erikl3@kth.se}
\title{El Gamal Mixnets and Implementation of a Verifier \\ SA104x Degree Project in Engineering Physics \\
KTH Royal Institute of Technology \\
School of Computer Science and Communication
Supervisor: Douglas Wikström}
\begin{document}

\maketitle
\tableofcontents

\setlength{\parindent}{0pt}
\setlength{\parskip}{10pt plus 1pt minus 1pt}

\pagebreak
\section{Introduction}
Why cryptography?

Classical vs Modern cryptography

Public key cryptosystem

Uses (Communication, Signatures?, Verification)

Mix networks

Verificatum Mixnet

Verification and its importance

Implementation


\section{El Gamal Cryptography}

\subsection{Definition}
The El-Gamal cryptosystem is defined over a group $G_q =
\left<g\right>$ of prime order $q$, generated by $g \in G_q$. A
private key $x \in \mathbb{Z}_q$ is chosen randomly and is used to
compute the public key $(g,y) \in G_q \times G_q$ where $y =
g^x$. 

Encryption of a plaintext $m \in G_q$ is done by choosing a random $s
\in Z_q$ and computing $(u,v) \in G_q \times G_q$ where $u = g^s$ and
$v = y^sm$. Decryption of a ciphertext $(u,v) \in G_q \times G_q$ is
achieved by using the private key $x$ to compute $m = u^{-x}v$.

\subsection{Security}
Let $b = g^a \in G_q$ where $a \in \mathbb{Z}_q$. Then $a$ is said to
be the discrete logarithm of $b$ in the group $G_q$. There is
currently no known efficient classical algorithm that given $(G_q, g,
b)$ is able to calculate $a$ in a reasonable amount of time
(polynomial time). The discrete logarithm problem is thus considered
to be a hard problem. (Källa)

The security of the El Gamal cryptosystem relies on the difficulty of
discrete logarithm in finite cyclic groups $G_q$. This means that the
El Gamal cryptosystem is secure as long as no one is able to compute
the discrete logarithm in $G_q$ efficiently. (Källa)

\subsection{Properties}
The El Gamal cryptosystem is a homomorphic cryptosystem. This 




Generalization

\section{Cryptographic Primitives}

PRGs
ROs

\section{Mix Networks}



\subsection{Overview}

Intuitiv beskrivning (gör bättre)

http://www.rsa.com/rsalabs/staff/bios/ajuels/publications/universal/Universal.pdf

One purpose of mix networks, or mixnets, is to provide untraceability
to its users. A mixnet may, for example, take as input a list of
encrypted messages of different origins. These messages pass through
the mixnet and is output decrypted and in a randomized order. This
property may be used to enable anonymous voting systems.

A reencryption mixnet consists of a number of servers which
sequentially process the messages and reencrypts the list of messages
and outputs them in a randomized order. After passing through all
servers, the list of ciphertexts is decrypted and the result is a list
of the messages in random order. It is impossible to deduce from where
each element came.

\subsection{El Gamal Mixnets}



\subsection{Operation}

\subsection{Verification}

\section{Specification/Documentation}

Vilken dokumentation har vi använt oss av?


\section{Implementation of the Verifier}

\subsection{General Design Choices}

Programming language?

Objects, UML


\subsection{Third Party Libraries}

\subsubsection{Arithmetic Library}
GMP
why?

\subsubsection{XML Parser}
RapidXML
why?

\subsubsection{Cryptographic Primitives}
OpenSSL
why?

\subsubsection{Testing}
Google Test
why?

\subsection{Math Library}

Hur och varför har vi gjort som vi gjort?

\subsection{Pseudorandom Generators and Random Oracles}

\subsection{Verifier}

\subsection{Tests}

\subsection{Performance}

Viktigt?

\section{Conclusion}
Kunde dokumentationen ha gjorts bättre?


\section{References}

\end{document}
